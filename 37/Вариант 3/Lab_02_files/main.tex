\documentclass{article}
\usepackage[utf8]{inputenc}
\usepackage[russian]{babel}
\usepackage{tikz}

\begin{document}

\begin{center}
\textbf{Блок-схема алгоритма сортировки Шелла}
\end{center}

\begin{center}
\vspace{1cm}

\begin{tikzpicture}[node distance=1.5cm]

% Определяем стили
\tikzstyle{startstop} = [rectangle, rounded corners, minimum width=3cm, minimum height=1cm, text centered, draw=black, fill=red!30]
\tikzstyle{process} = [rectangle, minimum width=3cm, minimum height=1cm, text centered, draw=black, fill=orange!30]
\tikzstyle{decision} = [diamond, minimum width=3cm, minimum height=1cm, text centered, draw=black, fill=green!30]
\tikzstyle{arrow} = [thick,->,>=stealth]

% Узлы блок-схемы (столбиком сверху вниз)
\node (start) [startstop] {Начало};
\node (computeH) [process, below of=start] {Вычислить h[1]...h[t]};
\node (initK) [process, below of=computeH] {k := t; 1; -1};
\node (kCheck) [decision, below of=initK] {hk:=h[k]};
\node (setHk) [process, below of=kCheck] {i:=hk+1; N; 1};
\node (initI) [process, below of=setHk] {x:=a[i]};
\node (iCheck) [decision, below of=initI] {j:=i};
\node (setX) [process, below of=iCheck] {j>hk и x<a[j-hk]};
\node (setJ) [process, below of=setX] {a[j]:=a[j-hk]};
\node (pCheck) [decision, below of=setJ] {j:=j-hk};
\node (shift) [process, below of=pCheck] {a[j]:=x};
\node (decJ) [process, below of=shift] {конец};

% Основные вертикальные стрелки
\draw [arrow] (start) -- (computeH);
\draw [arrow] (computeH) -- (initK);
\draw [arrow] (initK) -- (kCheck) node[right, pos=0.5, yshift=0.0cm] {да};
\draw [arrow] (kCheck) -- (setHk);
\draw [arrow] (setHk) -- (initI) node[right, pos=0.5, yshift=0.0cm] {да};;
\draw [arrow] (initI) -- (iCheck);
\draw [arrow] (iCheck) -- (setX);
\draw [arrow] (setX) -- (setJ) node[right, pos=0.5, yshift=0.0cm] {да};;
\draw [arrow] (setJ) -- (pCheck);
\draw [arrow] (pCheck) -- (shift);

% Дополнительные стрелки с подписями "да" в самом начале стрелки
% 1. От initK к decJ (в обход справа) - ДА
\draw [arrow] (initK.east) -- ++(3,0) node[right, yshift=0.2cm, pos=0.4] {нет} -- ++(0,-13.5) -- (decJ.east);

% 2. От shift к setHk (в обход слева) - ДА
\draw [arrow] (shift.west) -- ++(-2,0) node  -- ++(0,9) -- (setHk.west);

% 3. От pCheck к setX (петля слева) - ДА
\draw [arrow] (pCheck.west) -- ++(-1,0) node  -- ++(0,3) -- (setX.west);

% 4. От setX к shift (петля справа) - ДА
\draw [arrow] (setX.east) -- ++(1,0) node[right, yshift=0.2cm, pos=0.15] {нет} -- ++(0,-4.5) -- (shift.east);

% 5. От setHk к initK (петля справа) - ДА
\draw [arrow] (setHk.east) -- ++(1.5,0) node[right, yshift=0.2cm, pos=0.2] {нет} -- ++(0,2.5) -- (initK.east);

\end{tikzpicture}
\end{center}

\end{document}